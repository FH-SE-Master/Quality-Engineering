\section{Qualität}
\label{sec:quality}
In diesem Abschnitt werden die Produktqualität und die Qualität im Einsatz für die Internationalisierung diskutiert, wobei unter Internationalisierung das Einführen des Produkts in einen anderen Markt verstanden wird.
\newline
\newline
Als Beispiel für die Produktqualität werden Qualitätsmerkmale im Bezug auf die XML-Verarbeitung, die über die Anwendung \emph{clever\textbf{interface}} der Softwarelösung \emph{clevercure} erfolgen, definiert.
\subsection{Produktqualität nach ISO 25010}
\label{sec:qualtity-product-quality}
Dieser Abschnitt beinhaltet die tabellarische Darstellung der gewählten Merkmale Produktqualität.
\newline
\bgroup
\def\arraystretch{1.5}%  1 is the default, change whatever you need
\begin{tabularx}{\textwidth}{ p{70pt} | X | X | X }
	\hline
	& \textbf{Beschreibung} & \textbf{Maß} & \textbf{Messvorgang} \\ \hline
	& \textbf{Priorität} & \textbf{Maßeinheit} & \textbf{Schwellwert} \\ \hline
	\textbf{Funktionalität} & \multicolumn{3}{p{300pt}}{Anbinden des \emph{ERP}-Systems des Kunden an das \emph{clever\textbf{cure}}-System} \\ \hline
	\textbf{Merkmal} & \multicolumn{3}{X}{Leistungseffizienz} \\ \hline
	\textbf{Definition} & \multicolumn{3}{p{380pt}}{Einrichten des \emph{SAP}-Plugins beim Kunden} \\ \hline
	\textbf{Submerkmal} & \multicolumn{3}{X}{Zeitverhalten} \\ \hline 
	\textbf{Attribut} & Einrichtungsdauer & Ankunft beim Kunden bis Fertigstellung & Zeitmessung  \\ \hline
	               & Sehr Hoch               & Zeit in Minuten & $<= 18$  \\ 
    \hline 
    \hline 
	& \textbf{Beschreibung} & \textbf{Maß} & \textbf{Messvorgang} \\ \hline
	& \textbf{Priorität} & \textbf{Maßeinheit} & \textbf{Schwellwert} \\ \hline
	\textbf{Funktionalität} & \multicolumn{3}{p{300pt}}{Schulung der Key-User beim Kunden} \\ \hline
	\textbf{Merkmal} & \multicolumn{3}{X}{Benutzerfreundlichkeit} \\ \hline
	\textbf{Definition} & \multicolumn{3}{p{380pt}}{Schulung über die Kernfunktionalität von \emph{clever\textbf{cure}}} \\ \hline
	\textbf{Submerkmal} & \multicolumn{3}{X}{Erlernbarkeit} \\ \hline 
	\textbf{Attribut} & Testergebnisse & Testbewertung  & Testauswertung  \\ \hline
	& Hoch               & Prozent & $>= 80$  \\ \hline
\end{tabularx}
\egroup
\ \newline
\newline
\textbf{Begriffserklärungen:}
\begin{itemize}
	\item Das \emph{SAP-Plugin} wurde von einem Partner entwickeltet und integriert die Import- und Exportkomponenten von \emph{clever\textbf{interface}} in das \emph{ERP}-System des Kunden.
	\item Mit der \emph{Schulung über die  Kernfunktionalität von clever\textbf{cure}} sind folgende Bereiche mit inbegriffen.
	\begin{itemize}
		\item Die Lieferantenverwaltung,
		\item die Einkäuferverwaltung,
		\item das Vorlagenmanagement und
		\item das Workflowmanagement.
	\end{itemize}
\end{itemize}
\newpage
\subsection{Qualität im Einsatz nach ISO 25010}
\label{sec:qualtity-quality-in-use}
Dieser Abschnitt beinhaltet die tabellarische Darstellung der gewählten Merkmale Qualität im Einsatz.
\newline
\newline
\bgroup
\def\arraystretch{1.5}%  1 is the default, change whatever you need
\begin{tabularx}{\textwidth}{ p{70pt} | X | X | X }
	\hline
	& \textbf{Beschreibung} & \textbf{Maß} & \textbf{Messvorgang} \\ \hline
	& \textbf{Priorität} & \textbf{Maßeinheit} & \textbf{Schwellwert} \\ \hline
	\textbf{Funktionalität} & \multicolumn{3}{p{300pt}}{Anlegen eines Storage für einen Kunden für das Dokumentenmanagementsystems.} \\ \hline
	\textbf{Merkmal} & \multicolumn{3}{X}{Zufriedenheit} \\ \hline
	\textbf{Definition} & \multicolumn{3}{p{380pt}}{Anlegen eines Storage im Mutterland des Kunden (Z.B. Amerika)} \\ \hline
	\textbf{Submerkmal} & \multicolumn{3}{X}{Vertrauen} \\ \hline 
	\textbf{Attribut} & Anlagedauer & Zeitraum & Zeitmessung \\ \hline
	& Hoch               & Zeit in Stunden & $<= 12$  \\ 
	\hline 
	\hline 
	& \textbf{Beschreibung} & \textbf{Maß} & \textbf{Messvorgang} \\ \hline
	& \textbf{Priorität} & \textbf{Maßeinheit} & \textbf{Schwellwert} \\ \hline
	\textbf{Funktionalität} & \multicolumn{3}{p{300pt}}{Backup der Kundenstorages} \\ \hline
	\textbf{Merkmal} & \multicolumn{3}{X}{Risikofreiheit} \\ \hline
	\textbf{Definition} & \multicolumn{3}{p{380pt}}{Periodisches inkrementelles Backup des Kundenstorages} \\ \hline
	\textbf{Submerkmal} & \multicolumn{3}{X}{Minderung von Wirtschaftsrisiken} \\ \hline 
	\textbf{Attribut} & Dauer des Backups  & Zeitraum  & Zeitmessung  \\ \hline
	& Sehr Hoch               & Zeit in Minuten & $<= 15$  \\ \hline
\end{tabularx}
\egroup
\ \newline 
\newline
\textbf{Begriffserklärungen:}
\begin{itemize}
	\item Ein \emph{Storage} ist ein persistenter Speicherbereich eines Kunden, der alle anfallenden Dokumente von \emph{clever\textbf{cure}} wie Bestellbestätigung, Gutschrift usw. enthält.
\end{itemize}
\ \newpage
\subsection{Qualitätssicherung nach ISO 250xx}
\label{sec:qualtity-quality-ensurement}
In diesem Abschnitt wird die Qualitätssicherung nach ISO 250xx diskutiert.
\subsubsection{Leistungseffizienz}
\label{sec:quality-qualtity-ensurement-openTrans}
Im folgenden werden die Maßnahmen für die Qualitätssicherung der Anwendung \emph{clever\textbf{interface}} der Softwarelösung \emph{clever\textbf{cure}} angeführt.
\newline
\newline
Die \emph{Service-Provider}, bei denen unsere \emph{Services} gehostet werden, müssen monatliche Reporte, die einen zeitlichen Verlauf der Ressourcenauslastung der Prozessoren und Speicherauslastung enthalten, zur Verfügung stellen, die von den IT-Verantwortlichen monatlich ausgewertet werden, um auf ein Ansteigen des Ressourcenbedarfs entweder durch die Optimierung der Softwarekomponente oder einem Aufstocken der Systemressourcen entgegen wirken zu können.
\newline
\newline
Es werden die Schwellwerte $80\%$ Prozessorauslastung über einen Zeitraum von 30 Minuten und $75\%$ Auslastung des Arbeitsspeichers festgelegt, bei denen die \emph{Service-Provider} innerhalb von 5 Minuten eine Alarmmeldung an angegebene Kontaktadressen in Form von E-Mail verschicken müssen. 
\newline
\newline
Den einzelnen Kunden und Lieferanten werden nur begrenzte Systemressourcen zur Verfügung gestellt, wobei eine Überbuchung der gesamten Systemressourcen von $30\%$ zulässig ist. 
\newline
\newline
Die Kunden müssen monatliche Reports ihrer genutzten Systemressourcen bekommen, wobei im Falle einer ungewöhnlichen hohen Nutzung Kontakt mit den Verantwortlichen des Kunden Kontakt aufgenommen werden muss, um festzustellen warum der Kunde einen erhöhten Ressourcenverbrauch aufweist. Im Falle einer ineffizienten Nutzung muss dem Kunden Hilfestellung angeboten werden um den Ressourcenverbrauch wieder auf ein normales Niveau zu bringen. Im Falle eines erhöhten Datenaufkommens muss der Kunde darauf hingewiesen werden, das ein erhöhen seiner Systemressourcen von Nöten sein wird um die Effizienz des Datenimport und -export zu gewährleisten zu können.  

\subsubsection{Kompatibilität}
\subsubsection{Zufriedenheit}
\subsubsection{Flexibilität}



