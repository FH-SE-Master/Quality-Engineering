\section{Qualität}
\label{sec:quality}
In diesem Abschnitt werden die Produktqualität und die Qualität im Einsatz für die Internationalisierung diskutiert, wobei unter Internationalisierung das Einführen des Produkts in einen anderen Markt verstanden wird.
\newline
\newline
Als Beispiel für die Produktqualität werden Qualitätsmerkmale im Bezug auf die XML-Verarbeitung, die über die Anwendung \emph{clever\textbf{interface}} der Softwarelösung \emph{clevercure} erfolgen, definiert.
\subsection{Produktqualität}
\label{sec:qualtity-product-quality}
Dieser Abschnitt beinhaltet die tabellarische Darstellung der gewählten Merkmale Produktqualität.
\newline
\bgroup
\def\arraystretch{1.5}%  1 is the default, change whatever you need
\begin{tabularx}{\textwidth}{ p{70pt} | X | X | X }
	\hline
	& \textbf{Beschreibung} & \textbf{Maß} & \textbf{Messvorgang} \\ \hline
	& \textbf{Priorität} & \textbf{Maßeinheit} & \textbf{Schwellwert} \\ \hline
	\textbf{Funktionalität} & \multicolumn{3}{p{300pt}}{Anbinden des \emph{ERP}-Systems des Kunden an das \emph{clever\textbf{cure}}-System} \\ \hline
	\textbf{Merkmal} & \multicolumn{3}{X}{Leistungseffizienz} \\ \hline
	\textbf{Definition} & \multicolumn{3}{p{380pt}}{Übersetzung der Protokollvorlagen} \\ \hline
	\textbf{Submerkmal} & \multicolumn{3}{X}{Zeitverhalten} \\ \hline 
	\textbf{Attribut} & Übersetzungsdauer & Übermittlung der deutschen Version an Übersetzer bis dessen Rücksendung & Zeitmessung  \\ \hline
	               & Sehr Hoch               & Zeit in Tagen & $<= 14$  \\ 
    \hline 
    \hline 
	& \textbf{Beschreibung} & \textbf{Maß} & \textbf{Messvorgang} \\ \hline
	& \textbf{Priorität} & \textbf{Maßeinheit} & \textbf{Schwellwert} \\ \hline
	\textbf{Funktionalität} & \multicolumn{3}{p{300pt}}{Schulung der Key-User beim Kunden} \\ \hline
	\textbf{Merkmal} & \multicolumn{3}{X}{Benutzerfreundlichkeit} \\ \hline
	\textbf{Definition} & \multicolumn{3}{p{380pt}}{Übersetzte Schulungsvideos für Key-User} \\ \hline
	\textbf{Submerkmal} & \multicolumn{3}{X}{Erlernbarkeit} \\ \hline 
	\textbf{Attribut} & Testergebnisse & Testergebnisse  & Testauswertung  \\ \hline
	& Hoch               & Prozent & $>= 80$  \\ \hline
\end{tabularx}
\egroup
\ \newline
\newline
\textbf{Begriffserklärungen:}
\begin{itemize}
	\item Beim Anbinden eines \emph{ERP}-Systems des Kunden werden Protokolle verwendet, welche die Nötigen Arbeiten definieren und deren Resultate und aktuellen Status beinhalten. 
	\item Es werden online Schulungsvideos für die Key-User angeboten, welche die Funktionalität, Einstellungsmöglichkeiten und Best Practise von mit dem Umgang mit \emph{clever\textbf{cure}} erklären.
\end{itemize}
\newpage
\subsection{Qualität im Einsatz}
\label{sec:qualtity-quality-in-use}
Dieser Abschnitt beinhaltet die tabellarische Darstellung der gewählten Merkmale Qualität im Einsatz.
\newline
\newline
\bgroup
\def\arraystretch{1.5}%  1 is the default, change whatever you need
\begin{tabularx}{\textwidth}{ p{70pt} | X | X | X }
	\hline
	& \textbf{Beschreibung} & \textbf{Maß} & \textbf{Messvorgang} \\ \hline
	& \textbf{Priorität} & \textbf{Maßeinheit} & \textbf{Schwellwert} \\ \hline
	\textbf{Funktionalität} & \multicolumn{3}{p{300pt}}{Anlegen eines Storage für einen Kunden für das Dokumentenmanagementsystems.} \\ \hline
	\textbf{Merkmal} & \multicolumn{3}{X}{Zufriedenheit} \\ \hline
	\textbf{Definition} & \multicolumn{3}{p{380pt}}{Anlegen eines Storage über einen webbasierten Wizard.} \\ \hline
	\textbf{Submerkmal} & \multicolumn{3}{X}{Vertrauen} \\ \hline 
	\textbf{Attribut} & Anlagedauer & Zeitraum & Zeitmessung \\ \hline
	& Hoch               & Zeit in Minuten & $<= 10$  \\ 
	\hline 
	\hline 
	& \textbf{Beschreibung} & \textbf{Maß} & \textbf{Messvorgang} \\ \hline
	& \textbf{Priorität} & \textbf{Maßeinheit} & \textbf{Schwellwert} \\ \hline
	\textbf{Funktionalität} & \multicolumn{3}{p{300pt}}{Reporte für den Administrator über ausgeführte administrative Aktionen} \\ \hline
	\textbf{Merkmal} & \multicolumn{3}{X}{Risikofreiheit} \\ \hline
	\textbf{Definition} & \multicolumn{3}{p{380pt}}{Zeitangabe in der Zeitzone des Administrators mit Angabe der Zeitzone in der die Aktion ausgeführt wurde} \\ \hline
	\textbf{Submerkmal} & \multicolumn{3}{p{380pt}}{Minderung von Wirtschaftsrisiken} \\ \hline 
	\textbf{Attribut} & Korrektheit der Umrechnung  & Wahrheitswert  & Zeitumrechnung  \\ \hline
	& Sehr Hoch               & Korrekt oder Falsch & Korrekt \\ \hline
\end{tabularx}
\egroup
\ \newline 
\newline
\textbf{Begriffserklärungen:}
\begin{itemize}
	\item Ein \emph{Storage} wird von einem Kunden über einen webbasierten \emph{Wizard} erstellt, der den Kunden durch die nötigen Einstellungen führt inklusive Erklärungen,  verlinkter Dokumentation und mit der Möglichkeit eines direkten Kontakt zum \emph{Support}.
	\item Die Administrator erhalten je nach Einstellung aber mindestens einmal pro Woche Reporte über durchgeführte administrative Tätigkeiten am System.
\end{itemize}



