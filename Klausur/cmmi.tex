\section{ISO 9000 - CMMI}
\label{sec:cmmi}
\subsection{Ermittlung der Fähigkeiten und des Reifegrads}
\subsubsection{Ursachenanalyse und Problemlösung}
\label{sec:cmmi-1}
\textbf{Zweck:}
\newline
Der Zweck ist es die Sprachfertigkeiten unserer \emph{Support}-Mitarbeiter bezüglich der englischen Sprache in Wort und Schrift zu ermitteln. Es soll ermittelt werden inwieweit die \emph{Support}-Mitarbeiter in der Lage sind mit amerikanischen Benutzern in Englischer Sprache Ursachenanalyse zu betreiben und Problemlösungen an den Benutzer zu kommunizieren.
\newline
\newline
\textbf{Einführende Hinweise:}
\newline
Die Evaluierungen der Sprachfertigkeiten der \emph{Support}-Mitarbeiter wird von einer externen Stelle durchgeführt, die in diesem Bereich fachkundig ist. Die zu evaluierenden Fertigkeiten sind wie folgt vorgegeben.
\begin{itemize}
	\item Aufnahme der allgemeinen \emph{Support}-Anfragen,
	\item die Kommunikation von Lösungen bekannter Probleme,
	\item schriftliche Kommunikation mit den Benutzern und
	\item die Art und Weise wie mit den Kunden kommuniziert wird.  
\end{itemize} 
\ \newline
Nachdem die Evaluierungen abgeschlossen sind
\newline
\newline
\textbf{Ziel: Ermittlung der Sprachfertigkeiten:}
\newline
Es ist das Ziel die Sprachfertigkeiten unserer \emph{Support}-Mitarbeiter in der Englischen Sprache in Wort und Schrift zu ermitteln. Der Fokus liegt dabei auf die Ursachenanalyse und die Kommunikation der Problemlösungen an die Benutzer der allgemeinen und häufigsten \emph{Support}-Anfragen.
\newline
\newline
\textbf{Praktik: Ermittlung der Sprachfertigkeiten:}
\newline
Die Evaluierungen der \emph{Support}-Mitarbeiter sollen einzeln erfolgen,  wobei die Evaluierung mittels einem Rollenspiels ermittelt werden soll. Der Auditor spielt dabei einen amerikanischen Kunden, der insgesamt vier \emph{Support}-Anfragen hat, wobei diese nicht exakt formuliert werden sollen. Der \emph{Support}-Mitarbeiter soll dazu angeregt werden das Problem zweimal mittels verbaler Kommunikation und zweimal mittels E-Mail-Kommunikation (mindestens zwei E-Mails) zu ermitteln hat.  

\subsubsection{Organisationsweites Training}
\label{sec:cmmi-2}
\textbf{Zweck:}
\newline
Der Zweck ist es die Fähigkeiten mit dem Umgang unserer Mitarbeiter mit Mitarbeitern der amerikanischen Kunden im direkten Umgang im Bezug auf \emph{cultural awareness} zu ermitteln. Damit soll ermittelt werden, welche Mitarbeiter in welchem Umfang Schulungen bezüglich \emph{cultural awareness} benötigen.
\newline
\newline
\textbf{Einführende Hinweise:}
\newline
Es sollen Evaluierungen mit einer externen Fachkraft im Bereich \emph{cultural awareness} durchgeführt werden, um die Fähigkeiten der Mitarbeiter mit dem Umgang mit amerikanischen Kunden zu ermitteln. Folgende Aspekte sollen evaluiert werden.
\begin{itemize}
	\item Die Fähigkeit zu diskutieren,
	\item die Fähigkeit zu argumentieren,
	\item die Fähigkeit Konfliktpotentiale zu erkennen und 
	\item die Fähigkeit des privaten Umgangs.
\end{itemize}
\ \newpage
\textbf{Ziel: Ermittlung der \emph{cultural awareness}:}
\newline
Das Ziel ist es zu ermitteln inwieweit unsere Mitarbeiter in der Lage sind mit amerikanischen Kunden in verschiedenen Situationen zu interagieren. Damit sollen Schwächen der Mitarbeiter in verschiedenen Situationen ermittelt werden, um die Schulungen so effizient wie möglich zu gestalten.
\newline
\newline
\textbf{Praktik: Ermittlung der \emph{cultural awareness}:}
\newline
Die Evaluierungen sollen in kleinen Gruppen bis maximal 5 Personen abgehalten werden. Mittels Rollenspiele sollen folgende Situationen durchgespielt werden, um die Fähigkeiten der Mitarbeiter in verschiedenen Situation zu ermitteln:
\begin{itemize}
	\item Eine Diskussion mit einem Kunden bezüglich eines Verständnisproblems des Kunden über eine Vertragsoption.
	\item Eine Argumentation warum eine Funktionalität der Software so realisiert wurde.
	\item Ein Gespräch dass sich zu einen Streitgespräch entwickelt, wobei der Mitarbeiter deeskalieren soll.
	\item Ein privates Gespräch mit einem Kunden verschiedener Hierarchieebenen.
\end{itemize}

\subsubsection{Risikomanagement}
\textbf{Zweck:}
\newline
Der Zweck ist es zu ermitteln ob die Risikoanalysen bezüglich dem amerikanischen Markt adäquat sind. Damit ist gemeint ob Risiken überschätzt oder sogar unterschätzt wurden.
\newline
\newline
\textbf{Einführende Hinweise:}
\newline
Die Risikoanalysen sollen unter Hilfenahme der Rechtsabteilung und wenn nötig externe Fachkräfte auf Korrektheit geprüft werden. 
\newline
\newline
\textbf{Ziel: Prüfung der Risikoanalysen auf Korrektheit:}
\newline
Das Ziel ist es zu Vermeiden das die Risikoanalysen Aufgrund fehlender Erfahrungen am amerikanischen Markt zu pessimistisch oder zu optimistisch durchgeführt wurden. Damit soll mehr Vertrauen in die Risikoanalysen geschaffen werden sowie möglichen Fehlbetrachtungen entgegengewirkt werden.
\newline
\newline
\textbf{Praktik: Prüfung der Risikoanalysen auf Korrektheit:}
\newline
Es sollen die Risikoanalysen von den erfahrensten Mitarbeitern des Projektmanagements, der Rechtsabteilung von mehreren Blickwinkeln nochmals diskutiert werden. Wird erkannt das es bei einzelnen Teilen der Analyse intern es zu wenig Erfahrung gibt, dann soll eine externe Fachkraft mit der nötigen Erfahrung hinzugezogen werden.
\newpage




 